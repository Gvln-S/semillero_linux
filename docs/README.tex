\documentclass[conference, a4paper]{IEEEtran}
\usepackage[utf8]{inputenc}
\usepackage[T1]{fontenc}
\usepackage[spanish]{babel}
\usepackage{amsmath}
\usepackage{graphicx}
\usepackage{url}
\usepackage{xcolor}

\begin{document}

\title{Propuesta de Semillero de Investigación: Asistente Inteligente para la Administración de Sistemas Linux}

\author{
    \IEEEauthorblockN{Luisa Fernanda Carpintero}
    \IEEEauthorblockA{Universidad de La Sabana\\
    Estudiante de Ingeniería Informática\\
    Email: luisacarga@unisabana.edu.co}

    \IEEEauthorblockN{Santiago Gavilán Páez}
    \IEEEauthorblockA{Universidad de La Sabana\\
    Estudiante de Ingeniería Informática\\
    Email: santiagogapa@unisabana.edu.co}
}

\maketitle

\begin{abstract}
Esta propuesta presenta la creación de un semillero de investigación bajo la rama CIS (Computational Intelligence Society) de IEEE, enfocado en el 
desarrollo de un asistente inteligente que facilite la administración de sistemas Linux para usuarios principiantes mediante el uso de procesamiento 
de lenguaje natural (NLP). El semillero tiene como objetivo introducir a los investigadores en los fundamentos de Linux y la inteligencia artificial, 
utilizando tecnologías gratuitas como Ubuntu y Python. El proyecto se desarrollará durante 16 semanas, del 4 de agosto al 24 de 
noviembre de 2025, con reuniones semanales de 2 horas los dias jueves de 4:00 a 6:00, culminando en un prototipo funcional. Esta iniciativa busca 
fomentar el aprendizaje práctico y atraer a estudiantes interesados en automatización e inteligencia artificial.
\end{abstract}

\begin{IEEEkeywords}
Linux, Inteligencia Artificial, Procesamiento de Lenguaje Natural, Semillero de Investigación, Automatización
\end{IEEEkeywords}

\section{Introducción}
La rama CIS (Computational Intelligence Society) de IEEE en la Universidad de La Sabana se centra en la automatización e inteligencia artificial aplicada 
a la tecnología. En este contexto, proponemos la creación de un semillero de investigación titulado ``Asistente Inteligente para la Administración de 
Sistemas Linux''. Este semillero busca proporcionar a los investigadores una introducción práctica a los sistemas Linux y a las técnicas de 
inteligencia artificial, específicamente procesamiento de lenguaje natural (NLP), a través de un proyecto colaborativo que resulte en un prototipo 
funcional.


\section{Relevancia del Proyecto}
El sistema operativo Linux es fundamental en áreas como servidores, computación en la nube, DevOps y ciberseguridad, siendo una herramienta esencial 
en la industria tecnológica. Sin embargo, su curva de aprendizaje puede ser intimidante para principiantes debido a su interfaz de línea de comandos. 
Por otro lado, la inteligencia artificial, especialmente el procesamiento de lenguaje natural, está transformando la forma en que interactuamos con 
la tecnología, permitiendo interfaces más intuitivas.

El semillero propone desarrollar un asistente inteligente que traduzca instrucciones en lenguaje natural. Este proyecto es ideal para:
\begin{itemize}
    \item \textbf{Introducir Linux}: Los investigadores aprenderán comandos básicos, estructura de directorios y gestión de permisos en Linux, 
                                     habilidades altamente demandadas.
    \item \textbf{Iniciarse en IA}: Los investigadores participantes explorarán conceptos de NLP y fortalecerán sus habilidades en Python, 
                                    una de las herramientas más usadas en inteligencia artificial.
    \item \textbf{Atracción para investigadores}: La creación de un producto tangible y útil (un asistente) motivará a estudiantes principiantes
                                                  a unirse al semillero.
\end{itemize}

\section{Tecnologías Propuestas}
\begin{itemize}
    \item \textbf{Linux (Ubuntu)}: Distribución inicial para principiantes, usada para aprender comandos y hospedar el asistente.
    \item \textbf{Python}: Lenguaje de programación para scripting y desarrollo del asistente, usando bibliotecas como:
        \begin{itemize}
            \item \texttt{subprocess}: Para ejecutar comandos Linux desde Python.
            \item \texttt{spaCy}: Biblioteca gratuita para procesamiento de lenguaje natural, ideal para interpretar instrucciones.
        \end{itemize}
    \item \textbf{VirtualBox/WSL}: Entornos gratuitos para configurar Linux en cualquier sistema operativo, asegurando que todos los estudiantes
                                   puedan participar.
    \item \textbf{Opcional - Tkinter}: Biblioteca de Python para crear una interfaz gráfica simple, si se desea una experiencia más amigable.
\end{itemize}

\section{Plan de Trabajo}
El proyecto se desarrollará del 4 de agosto al 24 de noviembre de 2025 (16 sesiones). El cronograma es el siguiente:

\begin{itemize}
    \item \textbf{Semanas 1-5 (7 de agosto - 4 de septiembre)}: \textbf{Introducción a Linux}
        \begin{itemize}
            \item Temas: Distribuciones Linux (Ubuntu, Debian), comandos básicos, estructura de directorios, sistema de permisos en Linux y generalidades.
            \item Actividades: Configuración de entornos Linux (VirtualBox/WSL), prácticas de comandos.
            \item Entregable: Guía colaborativa de comandos Linux.
        \end{itemize}
    \item \textbf{Semanas 6-8 (11 de septiembre - 25 de septiembre)}: \textbf{Fundamentos de Python y NLP}
        \begin{itemize}
            \item Temas: Python básico, uso de \texttt{subprocess} para ejecutar comandos, introducción a NLP con \texttt{spaCy}, mapeo de frases a 
                         comandos.
            \item Actividades: Creación de scripts que ejecuten comandos Linux desde Python, análisis de texto simple.
            \item Entregable: Script inicial que mapea frases a comandos.
        \end{itemize}
    \item \textbf{Semanas 9-12 (9 de octubre - 23 de octubre)}: \textbf{Desarrollo del Asistente}
        \begin{itemize}
            \item Temas: Diseño del asistente (interfaz en consola o gráfica -dependera de la semana en la que se este y de las ideas de los
                         investigadores- con \texttt{Tkinter}), integración de Ollama para NLP local, reglas para comandos comunes.
            \item Actividades: Implementación del prototipo, pruebas iniciales.
            \item Entregable: Prototipo beta del asistente.
        \end{itemize}
\end{itemize}

\begin{thebibliography}{1}
\bibitem{spacy} Explosion, ``spaCy: Industrial-Strength NLP,'' \url{https://spacy.io}.
\bibitem{linuxbible} Negus, C., ``Linux Bible,'' 10th ed., Wiley, 2020.
\end{thebibliography}

\end{document}
